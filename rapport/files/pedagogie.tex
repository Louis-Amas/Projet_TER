\section{Pedagogie}

\subsection{Récapitulatif des réunions}

Durant la période à plein de temps de mon TER, j'ai eu l'occasion de partager mon travail avec deux chercheurs, 
Dr Luigi Liquori (INRIA), Dr Daniel De Carvalho (Innopolis) et un doctorant Mr Mansur Khazeev (Innopolis).
Leurs domaines de recherche sont les langages de programmation, et ils s'intéressent aux technologies blockchain et
aux langages de programmations associés au développement des smarts contracts.
Un des objectifs de mon projet a été d'aider le doctorant à réduire le champ de recherche de sa thèse. Durant toutes
les semaines du mois à plein temps nous avions une réunion par semaine. Étrangement je me suis retrouvé au cœur de 
ces réunions. On m'a donné la parole durant minimum une heure par semaine et j'avais carte blanche ou presque
sur les sujets abordés.  Au début, cela a été plutôt difficile je n'avais pas l'habitude de parler si longtemps avec
des personnes plus sages que moi. De plus, les réunions étaient en anglais et ce n'est pas ma langue maternelle.

Finalement, j'ai énormément appris et je pense que j'ai pu aussi leur apprendre des choses. J'ai réalisé de la programmation
en direct de smart contract modifiable afin de montrer directement le côté technique. J'ai aussi réalisé des graphiques et des
PowerPoint afin d'expliquer mes propos. Transmettre mes recherches et les connaissances que j'ai pu acquérir est je trouve
très intéressant, j'ai pu mettre à l'épreuve ce que je savais ou ce que je pensais savoir aux avis de chercheurs expérimentés
et cela permet de se rendre compte de savoir si ma pensée était correcte.

\subsection{Mon organisation}

Ce projet a été réalisé durant la période de confinement suite aux mesures sanitaires contre le COVID-19. J'ai réalisé
tout le projet de mon domicile familial. Cela faisait plus de 4 ans que je n'y avais pas habité et vivre de nouveau
et surtout travailler entouré de ma famille a été un réel défi. Afin de ne pas perdre un rythme de travail je me suis imposé
de commencer tous les jours le travail à 9h. En général, j'avais trois tâches principales à faire:

\begin{itemize}
    \item Rechercher. J'ai réalisé des centaines de recherches sur internet, j'ai aussi lu beaucoup d'information
        sur Twitter et Reddit qui malgré leur statut de réseau social sont selon moi de bonne source d'information
        pour les sujets scientifiques modernes.
    \item Réaliser des prototypes. J'ai réalisé beaucoup de prototype afin de comprendre en profondeur le fonctionnement
        des techniques que j'ai pu trouver lors de mes recherches.
    \item Préparer des PowerPoint ou des graphiques. L'objectif est de présenter les recherches aux chercheurs et de simplifier
        les explications des prototypes.
\end{itemize}

J'ai dédié énormément de temps à ce projet est je suis fier de ce que j'ai pu réaliser. Ce projet à éveiller mon esprit et la recherche
et m'amènera peut-être vers une thèse.
