\section{Problèmes soulevés}

Pouvoir modifier dynamiquement le comportement d'un smart contract comporte des risques.
En effet, comme expliquer précédemment les utilisateurs ont confiance dans les smarts contracts
car, leurs comportements sont immuables. Il est donc primordial de trouver des solutions afin de garder
la confiance des utilisateurs. Pour cela il existe différentes type de méthode appelé gouvernance:

\subsection{Gouvernance avec un administrateur}

La première technique de gouvernance qui est aussi la plus simple est avec un administrateur.
En effet, à la création d'un smart contract on peut définir un administrateur qui à tout contrôle
sur le comportement du smart contract. Il peut ajouter, modifier ou supprimer du comportement.
Dans ce cas utiliser le smart contract revient à utiliser une alternative centralisée. On perd
l'avantage du décentralisé de la blockchain.

\subsection{Gouvernance avec un système de vote}

Un système de gouvernance avec un système de vote permet de garder l'aspect décentralisé d'un smart contract.
Dans ce type de système il est nécessaire définir un protocole de proposition et d'acceptation
d'un changement. Pour définir la valeur d'un vote on peut utiliser un smart contract implémentant un Token.
Un Token est défini dans l'environnement Ethereum par le la norme ERC-20. Cette norme définie qu'un Token
fongible (Un bien fongible est un bien qui se caractérise par son appartenance à un genre et non par 
une identité propre). Un Token peut donc s'apparenter à une monnaie. Un Token peut s'échanger ou être créé.

À la création d'un smart contract modifiable gouverné par un système de Vote, on peut imaginer générer une quantité
fixe de Token est les attribuer à un nombre restreint de parti prenante. Afin d'accepter une proposition on demande
aux personnes ayant des Tokens de voter. Le vote consiste à pour toutes les personnes ayant des Tokens de donner leurs
accord ou non accord (oui ou non) par rapport à une proposition. S'ils donnent leurs accords alors leur quantité 
de Token est ajouté au nombre de vote pour la proposition. Si le nombre de vote pour une proposition est supérieur ou
égal à la quantité totale * 0.8. Alors 80\% des personnes ayant le droit de vote sont d'accord pour la modification.
Le code permettant d'ajouter une proposition et de voter doit être immuable afin de protéger contre un vote
souhaitant modifier le système de vote.

Par exemple: deux entreprises créer un smart contract modifiable (basé sur la bibliothèque Diamond). À la création
elle génère 1000 Tokens (500 pour les deux entreprises). Afin d'ajouter, modifier ou supprimer 
du comportement au smart contract il est nécessaire d'obtenir l'accord de 80\% des tokens créer. 
L'entreprise 1 demande d'ajouter une fonctionnalité nommée "A". La proposition A reçoit directement le nombre de token
de l'entreprise 1 soit 500. Si l'entreprise 2 vote "oui" alors le total des votes sera de 1000 et vu que $1000 > 1000*0.8$
alors la proposition est accepté.

L'avantage de cette méthode est que l'on peut très facilement ajouter des acteurs.

Par exemple: L'entreprise 2 à un accord avec une nouvelle entreprise nommée "3". Cet accord consiste à obtenir
25\% des droits de vote sur le smart contract modifiable. Afin de réaliser cette action l'entreprise 2 envoi 250
token à l'entreprise 3.


Ce système permet une grande flexibilité est permet d'instaurer un vrai système démocratique au sein de la blockchain.
Cela permet de récupérer la confiance des utilisateurs car, toute modification doit être approuvé.

\subsection{Gouvernance avec un Multisig wallet}

