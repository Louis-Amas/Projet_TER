\section{Mes Objectifs et les scénarios}

\subsection{Objectifs}

Les objectifs de ce projet ont été les suivants et inspirés au papier scientifique \cite{OOPSLA} qui évoquait un langages de programmation object-based où les méthodes se pouvaient modifier (de façon sûre) par un simple appel de méthode (\emph{self-inflicted extension}):

\begin{itemize}
    \item Créer un prototype de modèle de smart contract modifiable.
    \item Comparaison de ma solution avec une bibliothèque populaire.
    \item Comparaison avec l'état de l'art standard.
\end{itemize}

Dès débuts du mois d'octobre, j'ai commencé à me renseigner sur les technologies présentées au chapitre
précédent. Suite à quelque réunion avec Mr. Luigi Liquori, Directeur de Recherche Inria, nous avions trouvé un chemin vers lequel je pouvais m'orienter afin de réaliser mon premier \emph{smart contract modifiable}.

Nous avons alors pris la décision de réaliser des prototypes intégralement par nous même.
L'objectif est ici d'apprendre les bases du développement et déploiement d'un tel contrat,
sans profiter d'aucune abstraction.

Une fois ce prototype terminé nous avions prévu de comparer cette solution a l'implémentation
de la bibliothèque \emph{OpenZeppelin} \cite{OpenZeppelin}. Cette bibliothèque étant l'un des standard de l'environnement
des smarts contract Ethereum.

Durant le mois de projet, j'ai aussi contacté l'équipe de développeurs d'Ethereum afin d'obtenir
des ressources sur le sujet. Cette équipe m'a répondu est m'a envoyé les dernières techniques du
domaine. Nous avons alors décidé d'ajouter comme objectif la comparaison avec des techniques les
plus avancés avec mon prototype et l'implémentation d'OpenZeppelin.

\subsection{Scénarios et \emph{Use Cases}}

\subsubsection{Scénario 1}

Un état souhaite passer au vote électronique, il décide d’utiliser les smart contrats afin de
décentraliser leur solution. L’objectif est de prouver aux électeurs que le vote n’est pas truqué. L’état va
alors payer une équipe d’ingénieur afin de faire un système de vote qui pourra durer dans le temps.
Néanmoins ils souhaitent aussi pouvoir modifier le mécanisme de vote au fur des années. C’est ici que
mon projet de recherche prend tout son sens, je conseille cet état fictif d’utiliser mes recherches afin
d’utiliser un smart contract modifiable. L’objectif est pouvoir changer le système de vote dans le futur
tout en utilisant la technologie blockchain.

\subsubsection{Scénario 2}

Deux entreprises (A et B) souhaitent passer un accord, elles décident de partager leurs services
de ressources humaines pour réduire les frais. Elles décident de stocker le nombre d’heures passés à
travailler pour une entreprise de façon décentralisée (Smart Contract). Avec cette solution l’entreprise B
pourra être sûre que l’entreprise A ne ment pas et vice-versa. De plus A aimerait ajouter une nouvelle
entreprise dans quelques années au système. 
Pour cela, le smart contract doit pouvoir être modifié pour ne pas forcer A et
B de déployer un nouveau contrat contenant C (la nouvelle entreprise).
