\section{Contexte technologique}

Statut: En cours

Sujets:
\begin{itemize}
    \item Blockchain
    \item Bitcoin
    \item Ethereum
    \item EVM
    \item Solidity
\end{itemize}


\subsection{Blockchain}

\subsubsection{Bitcoin}

Une (ou un) blockchain, ou chaîne de blocs  Techniquement, il s'agit d'une base de données distribuée dont les informations stockées sont vérifiées 
et groupées à intervalles de temps réguliers sous forme de bloc, formant ainsi une chaîne de blocs. L'ensemble est sécurisé par cryptographie. 
La chaîne de blocs est alors stocké sur de multiple machine en même temps que l'on appelle des nœuds. 
Cette chaîne permet de protéger contre la falsification et la modification de la base de données par les nœuds de stockage.
C'est donc un registre distribué et sécurisé de toutes les transactions effectuées depuis le démarrage du système réparti.

La première technologie déployant un tel système est le Bitcoin. Le Bitcoin utilise la blockchain afin de représenter un registre de compte
distribué. Les utilisateurs souhaitant utiliser cette cryptomonnaie génère une paire de clé asymétrique.

Pour rappel, les chiffrements asymétriques sont constitués de deux clés l'une publique disponible par tous le monde et l'une privée qui doit
rester secrète. Les méthodes de chiffrements à clé publique permettent e de rendre un message inintelligible par toute personne n'ayant pas la clé 
privée d'un message chiffré par la clé publique complémentaire. Cette technique de chiffrement permet aussi de vérifier (Signer numériquement) qu'un
message a bien été écrit par quelqu'un possédant une paire de clé (Privée, publique).

Bitcoin utilise cette fonctionnalité afin d'autoriser une personne à échanger son argent. Un bloc de la blockchain Bitcoin contient des transactions
(des échanges de jeton BTC) signer par la clé privée de l'utilisateur souhaitant transférer ses jetons. Bitcoin calcul le hash de toutes ses 
transactions et crée un bloc à l'aide de la concaténation de tous les hashs de toutes les transactions ainsi que le hash du bloc précédent.

Cela permet de rendre quasi impossible la falsification et la modification de la chaîne. En effet, un attaquant voulant modifier la chaîne serait
détecté car, le hash de son bloc serait invalide s'il modifie même qu'une transaction. Le Bitcoin réalise l'exploit de créer un équivalent a l'argent
liquide mais numériquement.

\subsubsection{Ethereum}

Ethereum est une technologie qui révolutionne la blockchain en y ajoutant une fonctionnalité très intéressante nommé les smarts contract. Cette
technologie est toujours basée sur le même principe que Bitcoin afin de valider ses transactions mais ajoute quelques nouvelles fonctionnalités.

\subsection{Smart contract}

Les contrats intelligents ou smart contracts sont des protocoles permettant d'exécuter du code de manière distribuée dans un environnement blockchain.
Ce mécanisme permet de s'affranchir de l'architecture client / serveur mais aussi, des architectures distribuées traditionnels. Les développeurs de
tel smart contract peuvent assumer que leurs contract une fois déployé sur la blockchain est immuable et décentralisé.

L'immuabilité de tel contrat permet de gagner la confiance des utilisateurs. Un utilisateur peut regarder le code du contrat déployé est être sûr 
de son comportement. Cela permet par exemple de programmer des jeux d'argents de manière numérique sans tierce de confiance mathématiquement 
vérifiable juste.

\subsubsection{EVM}

Pour créer une tel technologie Ethereum utilise l'Ethereum Virtual Machine (EVM) cette machine virtuelle peut être comparé à la JVM mais de manière distribuée.
L'EVM est une machine basée sur la pile distribuée.
L'EVM est exécutée sur chaque nœud du réseau Ethereum. Et chaque exécution est signé de la même manière que Bitcoin afin de protéger contre la falsification
ou la modification d'une exécution.

\subsubsection{Solidity}

Solidity est un langage de programmation orienté objet dédié à l'écriture de contrats intelligents. Il est utilisé pour implémenter des smarts contrat sur diverses 
blockchains, notamment Ethereum. Solidity est un langage inspiré des langages orientées objets ainsi que du langage Javascript. L'objectif est de rendre simple et
compréhensible le développement de smart contracts à tous. Néanmoins sa simplicité peut causer des problèmes car, il n'y a pas de moyen simple de vérifier 
l'exactitude du programme. À la différence d'un langage fonctionnel. Sachant que les smarts contracts représentent souvent de l'argent il est très important
de faire attention au code déployé.
