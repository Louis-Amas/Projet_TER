\section{Introduction}

Tout d’abord, il est nécessaire d’expliquer le concept des technologies concernées. La
blockchain est une technologie permettant de faire la représentation d’un registre de compte de
manière numérique et distribué. Cela permet de s’affranchir d’un tiers de confiance, la première
technologie réalisant ces objectifs est le Bitcoin. 

Le Bitcoin est une monnaie décentralisée basé sur
la blockchain. C'est un registre de compte contenant toutes les transactions effectuées. Pour sécuriser
les accès à ce registre (permettre au ayant droit d’un compte de dépenser son argent), Bitcoin
combine deux principes fondamentaux, la cryptographie à clé publique ainsi que la signature
numérique, La cryptographie à clé publique permet de vérifier que un utilisateur possède réellement
un compte. La signature numérique permet de vérifier l’intégrité des transactions. Les transactions
sont des éléments d’un bloc et les blocs sont chaînés (mis les uns après les autres) à l’aide de la
signature numérique. Ce principe de chaîne permet d’assurer que si l’une des transactions est
modifiée alors la signature est modifiée. Une personne malveillante modifiant une transaction déjà
inscrite dans la chaîne de bloc sera détectée.

Des améliorations à ce principe de blockchain on était faite et il existe maintenant une
technologie appelée "smart contract". Cette technologie est un moyen d’exécuter un programme
de manière distribuée, permettant de s’affranchir d’un tiers de confiance. Cela peut être utilisé pour
réaliser par exemple des systèmes de vote sans possibilité de triche ou de modification du résultat
ou bien représenter un jeu d’argent (poker, paris sportif...). Cette technologie est enregistrée de la
même manière dans un registre non modifiable. L’ojectif de ce projet et de rechercher les manières
de pouvoir modifier le programme exécuter par un smart contract afin d’adapter le comportement
dynamiquement selon les besoins.
