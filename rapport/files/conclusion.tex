\section{Conclusion}

Ce projet m'a permis de découvrir un domaine qui m'intéresse vraiment, la blockchain. La blockchain est selon moi
le futur pour énormément d'applications centralisé. Et pouvoir travailler un mois à plein temps dessus m'a permis 
d'acquérir des compétences dans ce domaine prometteur. De plus, avoir été encadrer par des chercheurs de l'INRIA 
m'a permis de découvrir est m'intéresser au monde de la recherche que je trouve passionnant. J'ai pu avec leurs
aide acquérir et comprendre l'état de l'art dans le domaine des smarts contract modifiable et j'en suis à la fois
fier et excité. Nous avons pu définir des idées d'améliorations qui permettraient de simplifier la vitesse d'adoption
de cette technologie en la rendant plus sûr et moins difficile d'utilisations à tout les développeurs. J'ai
envie de continuer ma vie dans ce domaine qui est la blockchain et je sais que ce TER m'a aider à renforcer mes
connaissances. 

Ce fût une expérience humaine et qui malgré le contexte sanitaires difficile m'a permis de me rapprocher de mon 
professeur Dr Luigi Liquori qui m'a fait confiance et qui m'a mit en avant devant ses pairs chercheurs et doctorants.
J'espère après mon stage poursuivre mon travail dans ce domaine et j'espère pouvoir apporter quelque chose dans quelques
années à la communauté international. Après ce projet, je continuerai peut-être en thèse ou en entrepreneuriat. 

Je remercies Dr Luigi Liquori pour m'avoir fais confiance, je remercies aussi Dr Daniel Da Carvalho et Mr Manssur Khazeev
de m'avoir écouter et corriger mes travaux et je remercies toutes l'équipe pédagogique d'avoir pu me transformer en le 
futur diplômé que je serais très bientôt. 
